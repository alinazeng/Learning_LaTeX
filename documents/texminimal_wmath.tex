\documentclass[11pt,letter]{article}
\usepackage[top=1.00in, bottom=1.0in, left=1.1in, right=1.1in]{geometry}
\renewcommand{\baselinestretch}{1.1}
\usepackage{graphicx}
\usepackage{natbib}
\usepackage{amsmath}

\def\labelitemi{--}
\parindent=0pt

\begin{document}
\bibliographystyle{/Users/Lizzie/Documents/EndnoteRelated/Bibtex/styles/besjournals}
\renewcommand{\refname}{\CHead{}}

\begin{align}
N_{i}(t+1) & =
s_{i}(N_{i}(t)(1-g_{i}(t))+\phi_{i}B(t+\delta)
\end{align}
The production of new biomass each season follows a basic R* competition model: new biomass production depends on its resource uptake ($f_i(R)$ converted into biomass at rate $c_i$) less maintenance costs ($m_i$), with uptake controlled by $a_i$ and $u_i$:
\begin{align}
\frac{\partial B_{i}}{\partial t} &  = [c_{i}f_{i}(R) - m_{i}]B_{i} \\
f_{i}(R) & = \frac{a_{i}R^{\theta_{i}}}{1+a_{i}u_{i}R^{\theta_{i}}}
\end{align}
With the initial condition:
\begin{align}
B(t+0) & = N_{i}(t)g_{i}(t)b_{0,i}
\end{align}
The resource ($R$) itself declines across a growing season due to uptake by all species and abiotic loss ($\epsilon$):
\begin{align}
\frac{\mathrm{d}R}{\mathrm{d}t} & = - \sum_{i=1}^{n}f_{i}(R)B_{i} -\epsilon R
\end{align}

\end{document}