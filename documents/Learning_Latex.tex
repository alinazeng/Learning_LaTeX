\documentclass[a4paper, 12pt]{article}
\usepackage{amsmath}
\usepackage{amsmath}
% define the title
\author{Alina Zeng}
\title{Don't Know What To Name This}

% update May-28, 2021
% taking tutorial here at https://latex-tutorial.com/tutorials/first-document/




\begin{document}
% generates the title on a new page
\pagenumbering{gobble}   % to hide the page number on the cover page
\maketitle
\newpage
\pagenumbering{arabic}   % to unhide the page number


% the section commands are numbered and will appear in the table of contents of your document. Paragraphs aren’t numbered and won’t show in the table of contents
% insert the table of contents
\tableofcontents
\newpage  % making a new page for contents

\section{For Forever}
Dear Evan Hansen, today is going to be a great day and here is why. \newline
You can \textsl{lean} on me!
You can \textbf{count} on me~ \newline


\subsection{All we see is light, for forever}
Excited for Hamilton Tour in 2022. \newline

\subsubsection {{\small Small} is {\Large big}}
What's cookin'

\paragraph {What to do when you feel hungry}

\subparagraph {Time to eat out!} Have not had a chance in so long.

\section{Math Equations}
\paragraph {A few simple ones}
\begin{equation*}
  f(x) = x^2
\end{equation*}
% The automatic numbering is a useful feature, but sometimes it’s necessary to remove them for auxiliary calculations. LaTeX doesn’t allow this by default, now we want to include a package that does

\begin{equation*}
  1 + 2 = 3 
\end{equation*}

\begin{equation*}
  1 = 3 - 2
\end{equation*}

\subsection{Alignment}
\paragraph{trying out alignment} ``=''
\begin{align*}
    f(x) &= x^2\\
    1 + 2 &= 3\\
  1 &= 3 - 2
\end{align*}
\subparagraph{LOL I don't quite like how this looks.}

\paragraph{trying out alignment} ``2''
\begin{align*}
    1 + &2 = 3\\
  1 = 3 - &2
\end{align*}
\subparagraph{This is a bit funky.}


\subsection{Would you like some more}
\paragraph{some simple LaTeX math functions}
\begin{align*}
  f(x) &= x^2\\
  g(x) &= \frac{1}{x}\\
  y(x) &= \left(\frac{1}{\sqrt{x}}\right)\\
  F(x) &= \int^a_b \frac{1}{3}x^3
\end{align*}
\subparagraph{more sophisticated functions can happen by combining various commands}



\subsection{trying out matrices}
\subparagraph{matrices inside parentheses}
\begin{equation*}
A = 
\begin{pmatrix}
1 & 2 & 3 \\
4 & 5 & 6 \\
7 & 8 & 9
\end{pmatrix}
\end{equation*}

\subparagraph{matrices without brackets}
\begin{equation*}
   \begin{matrix} 
   a_{11} & a_{12} & a_{13}  \\
   a_{21} & a_{22} & a_{23}  \\
   a_{31} & a_{32} & a_{33}  \\
   \end{matrix} 
\end{equation*}

\subparagraph{matrices have to happen within the equation environment}
\begin{equation*}
\begin{matrix}
1 & 0\\
0 & 1
\end{matrix}
\end{equation*}

\subparagraph{some more varieties}
\begin{equation*}
   \begin{vmatrix} 
   a_{11} & a_{12} & a_{13}  \\
   a_{21} & a_{22} & a_{23}  \\
   a_{31} & a_{32} & a_{33}  \\
   \end{vmatrix} 
\end{equation*}

\subparagraph{Here are examples with matrix 2x2 with pmatrix, bmatrix, vmatrix, Vmatrix environments:}
\begin{equation*}
\begin{matrix} 
a & b \\
c & d 
\end{matrix}
\quad  % side by side
\begin{pmatrix} 
a & b \\
c & d 
\end{pmatrix}
\quad
\begin{bmatrix} 
a & b \\
c & d 
\end{bmatrix}
\quad
\begin{vmatrix} 
a & b \\
c & d 
\end{vmatrix}
\quad
\begin{Vmatrix} 
a & b \\
c & d 
\end{Vmatrix}
\end{equation*}

\subparagraph{Small matrix environment} For more, refer to \textsl{https://www.math-linux.com/latex-26/faq/latex-faq/article/how-to-write-matrices-in-latex-matrix-pmatrix-bmatrix-vmatrix-vmatrix} \\
\begin{center}
\textbf{I love small matrices such as$\big(\begin{smallmatrix} a & b\\ c & d \end{smallmatrix}\big)$}
\end{center}


\section{Ending}
\ldots{} and here it ends.

\paragraph{will continue tmrw} at https://latex-tutorial.com/tutorials/amsmath/
\end{document}
